\section{Uma breve introdução à extensão espaço-tempo-simétrica da mecânica quântica}
\label{objetivos}

Na MQ, o tempo é um parâmetro contínuo que é usado para rotular a solução da equação de onda, já a posição de uma partícula é um operador, e portanto seu valor sob uma medição é probabilístico. Essa assimetria é às vezes atribuida ao caráter não relativístico da equação de Schrödinger. Embora parcialmente correto, este argumento é insuficiente para justificar toda a disparidade entre espaço e tempo no formalismo da MQ.

Normalmente para descrever a probabilidade de encontrar uma partícula nos referimos à região $x$ no instante $t$, porém não temos a capacidade de prever o instante exato de uma provável medição. Nesse contexto, a ideia do formalismo espaço-tempo-simétrico foi pensada inicialmente para apresentar uma forma do espaço e o tempo desempenharem papéis equivalentes, possibilitando uma descrição probabilística tanto da posição $x$ como do instante de tempo $t$ em que observamos uma partícula.

Sabemos que $\psi(x| t) = \langle x|\psi(t)\rangle$ nos dá a amplitude de probabilidade de encontrar a partícula em $x$, dado que o tempo de detecção é $t$. Providos da teoria STS que será formulada detalhadamente no Cap. 2, vamos poder nos perguntar sobre a probabilidade de encontrar a partícula entre $[x, x + dx]$ e $[t, t + dt]$. Nesse cenário, indagar sobre o estado de uma partícula em um determinado momento $t$ não faz sentido, assim como perguntar na MQ usual sobre o estado dessa partícula em uma dada posição $x$. Veremos que para manter a simetria aqui proposta, precisamos de uma função de onda do tipo $\pmb{\phi}(t| x) = \langle t|\pmb{\phi}(x)\rangle$, onde $x$ agora é um parâmetro contínuo e $t$ é o autovalor de um observável. $|\psi(t|x)|^2$ nos fornece o TOA ideal na posição $x$.

Atualmente, a extensão STS foi aplicada na Ref. \cite{Parana}, no calculo do tempo de travessia de uma barreira de potencial. Os autores resolveram analíticamente a equação "dinâmica" da extensão STS no limite de potenciais $V(x) \ll E$ e  $V(x) \gg E$, sendo $E$ a energia da partícula. Foi mostrado que o valor esperado do tempo de chegada no regime de tunelamento tem a forma de uma média de energia dos tempos clássicos adicionado de uma contribuição quântica. Em outro estudo, \cite{Ricardo}, a teoria é também utilizada para prever tempos de travessia de uma barreira de potencial, mas aqui aplicada a um experimento eletromagnético que simula o tunelamento quântico. Essa revisão demonstrou uma melhor concordância com os experimentos em comparação com os modelos de Büttiker-Landauer e o \textit{phase-time}.