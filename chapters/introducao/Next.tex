\section{O que vem a seguir}
\label{proxcaps}
Agora, vejamos o que abordaremos nos próximos capítulos.


No \textbf{capítulo \ref{chap:cap2}}, apresentaremos uma revisão mais detalhada do formalismo STS e de outras distribuições temporais para o tempo de chegada: a distribuição axiomática de Kijowski e o fluxo quântico.

\begin{itemize}
  \item Seção \ref{sec:cap2intro} Revisar a distribuição de Kijowski e o fluxo quântico.  
  \item Seção \ref{sec:stsreview}: Revisar todos os aspectos matemáticos do formalismo STS.
\end{itemize}



A partir do \textbf{capítulo \ref{chap:cap3}} apresentaremos nossos resultados originais.

\begin{itemize}
  \item Seção \ref{sec:bases}: Vamos estabelecer um paralelo entre $|\psi(t) \rangle$ e $|\pmb{\phi}(x)\rangle$ nas bases de energia e momento, respectivamente.
  \item Seção \ref{sec:interpretando}: Aqui, vamos interpretar mais precisamente o formalismo STS, e discutir sua conexão com a MQ usual.

Para o \textbf{capítulo \ref{chap:cap4}}, iremos aplicar nossos resultados a problemas específicos.
 
  \item Seção \ref{sec:solVarb}: Vamos resolver a equação "dinâmica" \text{ }para o caso de um potencial $V = V(x)$  qualquer.
  \item Seção \ref{sec:comparandosol}: Por fim, vamos comparar a solução da Sec. \ref{sec:solVarb} com o modelo de Kijowski para uma partícula que atravessa uma barreira de potencial. Discutiremos a possibilidade de estender a teoria STS com o intuito de levar em consideração a interferência entre momentos positivos e negativos. 
\end{itemize}

No \textbf{capítulo \ref{chap:conclusao}}, concluímos revisando os resultados originais dessa dissertação e discutindo perspectivas de trabalhos futuros. 