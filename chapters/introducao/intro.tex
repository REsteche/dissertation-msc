\chapter{Introdução}
\label{chap:intro}

 \begin{citacao}
 “Time is what we want most, but what we use worst."- Willian Penn  
 \end{citacao}

	Ainda nos dias de hoje, tempo na mecânica quântica é considerado um tema controverso, e isso se deve em parte ao fato de que existem dois tratamentos possíveis para o mesmo. Por um lado, na equação de Schrödinger, tempo não é um observável (tal qual o comportamento atribuído à posição), mas sim um parâmetro externo. O estado de um sistema quântico e a distribuição de probabilidade de uma quantidade física são condicionados em um dado instante de tempo. Por outro lado, o tempo pode ser uma quantidade mensurável na  MQ. Podemos medir o instante de tempo em que uma quantidade física assume um certo valor inicial e, enquanto o sistema evolui, medir o outro instante em que a mesma quantidade assume um dado valor final. Exemplos disso são as medições de tempo de chegada, tempo de permanência, tempo de vida, entre outros. 

	Como um instante de tempo pode ser um observável, naturalmente levanta a questão de saber se podemos incluí-lo na formulação ortodoxa da mecânica quântica. A maneira mais comum de incorporar quantidades clássicas na formulação quântica é o método de quantização canônica, que consiste em substituir os colchetes de Poisson de um par de variáveis canônicas pelos colchetes de comutação dos operadores correspondentes. Dessa forma, dado o hamiltoniano $H(q, p)$ de um sistema clássico conservativo (com nenhuma dependência temporal explícita), sempre podemos fazer uma transformação canônica de $(q, p)$ para novas variáveis canônicas $(H, T)$, onde $H$ é o hamiltoniano do sistema e $T$ sua variável conjugada, que satisfaça a equação de Hamilton

\begin{equation} \label{eq1.1}
	\frac{dT}{dt} = \{ H, T \} = \frac{\delta H}{\delta H} \frac{\delta T}{\delta T} - \frac{\delta T}{\delta H} \frac{\delta H}{\delta T} = 1.
\end{equation}

Na equação acima, perceba $T$ como um intervalo de tempo. Tomando então a quantização $ \{ H, T \} \Rightarrow 1/i \hbar [\hat{H},\hat{T}]$, onde postulamos que $\hat{H}$ e $\hat{T}$ são operadores auto-adjuntos, chegamos a um operador tempo satisfazendo a relação de comutação quântica,
\begin{equation} \label{eq1.2}
	 [\hat{H},\hat{T}] = i \hbar.
\end{equation}
Isso pode ser feito tanto na representação de Heisenberg quanto na de Schrödinger. A Eq.~(\ref{eq1.2}) nos leva à relação de incerteza,
\begin{equation} \label{eq1.3}
	 \Delta H \Delta T \ge  \frac{1}{2} \left| \left<   [\hat{H},\hat{T}]  \right> \right|. 
\end{equation}
onde $\Delta H$ e $\Delta T$ são os desvios quadráticos médios de $\hat{H}$ e $\hat{T}$ respectivamente.
    
     Contudo, já no início da construção teórica da MQ, Pauli demonstrou que a existência de um operador tempo auto-adjunto seguindo a Eq.~(\ref{eq1.2}) é incompatível com o caráter limitado ou semilimitado do espectro do hamiltoniano, como mostraremos em detalhes na seção a seguir. Em contraste à teoria quântica tradicional, que não consegue satisfazer a hermiticidade de um operador temporal e atender a relação de comutação acima apresentada, os autores na Ref. \cite{Dias} propuseram uma extensão espaço-tempo-simétrica da mecânica quântica como solução para esse problema. Nesse contexto, seguindo um processo de quantização distinto, é introduzido um operador tempo que satisfaz simultaneamente tanto a condição de hermiticidade como a de comutação canônica com um novo operador hamiltoniano.
    
    Nesta dissertação, nossa intenção será propor uma interpretação mais precisa para a extensão proposta na Ref. \cite{Dias}, comparar as suas previsões com as da MQ usual e investigar medições do tempo de travessia. Como ainda será discutido nessa introdução, o caráter condicional no tempo da mecânica quântica tradicional gera complicações nas previsões dos tempos de chegada e travessia. Na próxima seção, vamos dicutir com mais detalhes o argumento proposto por Pauli para contextualizar melhor o problema do tempo como um observável. Em seguida, vamos revisar alguns modelos existentes na literatura para tempos de chegada e de tunelamento. Esses modelos servirão para comparar com as previsões da \textit{space-time-symmetrical} (STS). 





 
 