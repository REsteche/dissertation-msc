\section{Os problemas do tempo de chegada e travessia}
\label{problems}

Muito tempo se passou após a construção do argumento de Pauli e o entendimento equivocado da comunidade científica de que a busca por operadores tempo seria fútil. Apenas na década de 1960 essa questão foi investigada novamente, e desde o inicio de 1990 esse assunto vem recebendo crescente atenção (\cite{arahabom}, \cite{allcock}, \cite{13}, \cite{MugaComplex}, \cite{halli}, \cite{livrotime1}, \cite{livrotime2}).


Atualmente, a pesquisa nessa área tem uma ênfase em resolver problemas práticos, como o aparecimento "instantâneo" $\text{ }$ de elétrons no efeito fotoelétrico (atualmente processos com duração menor que nanossegundos são considerados instantâneos  \cite{20}) e o decaimento de elementos metaestáveis. Essa área também inclui o estudo teórico do papel do tempo na mecância quântica  \cite{13}, bem como soluções gerais para o problema do tempo de chegada ou tempo de tunelamento. Almejando aprofundar um pouco mais nos problemas que serão abordados nessa dissertação, as seções seguintes serão dedicadas a realizar uma breve introdução sobre esses dois últimos problemas.


\subsection{O problema do tempo de chegada}
\label{arrival}

Considere um pacote de onda inicial $\psi (x|t_0)$ restrito à região $x < 0$. O problema do tempo de chegada consiste em prever a distribuição de probabilidade temporal para a partícula chegar à uma determinada posição $x > 0$. Em outras palavras, responder a pergunta \textit{"Qual é a probabilidade de uma partícula entrar em uma região do espaço pela primeira vez durante um determinado intervalo de tempo?”}. As abordagens existentes para este problema podem ser divididas em duas classes: (i) independente e (ii) dependente do dispositivo de medição. As distribuições da categoria (i) são intrínsecas ao estado da partícula, por isso são chamados de \textit{Ideal Time-of-Arival} (TOA ideal) (\cite{Delgado}, \cite{4}, \cite{Das2}). 

Uma abordagem comum para este problema assume que, como qualquer observável na MQ, o TOA ideal é determinado pela decomposição espectral de um operador auto-adjunto (um operador tempo), a condição inicial $\psi (x|t_0)$ e um hamiltoniano independente do aparato de medida. Por causa da objeção de Pauli, esses modelos ideais são compelidos a abdicar ou da relação de comutação canônica com o hamiltoniano ou da auto-adjunção do operador tempo \cite{livrotime1}. Dessa forma, a descrição teórica dos tempos de chegada, travessia e tunelamento permanece controversa, com inúmeras abordagens surgindo nas últimas décadas (\cite{allcock}, \cite{arahabom}, \cite{Ricardo}, \cite{when}).

Outros dois métodos tradicionais para problema do TOA ideal são a distribuição de probabilidade axiomática de Kijoswiski \cite{7} e a densidade de fluxo quântico (corrente de probabilidade) (\cite{8}, \cite{Das2}). No entanto, semelhante à abordagem que utiliza um operador tempo, essas duas formulações não são irrestritamente válidas. Por exemplo, sabe-se que o fluxo quântico pode prever probabilidades negativas, um efeito chamado refluxo quântico (\cite{Das}, \cite{9}). Falhas na distribuição de Kijowski para algumas configurações experimentais podem ser encontradas nas Refs. (\cite{when},\cite{13},\cite{14}). 

A classe (ii) de abordagens para o TOA modela o dispositivo de medição levando em consideração o fato de que, na prática, os detectores não apresentam um desempenho ideal como descrito acima. Existem inúmeras descrições teóricas do aparato de medição, por exemplo, usando potenciais complexos (\cite{MugaComplex},\cite{14},\cite{18}), o colapso da função de onda (\cite{18},\cite{20}), relógios quânticos \cite{Damborenea}, integrais de caminho (\cite{Schuss},\cite{Schuss2},\cite{Schuss3}), e o formalismo de Page e Wooters para condições de contorno absorventes \cite{pagewootters}. Vale ressaltar que esses métodos operacionais e as distribuições ideais tornam-se equivalentes em circunstâncias específicas distintas. Por exemplo, ao usar um potencial complexo que absorve idealmente a função de onda sem reflexão \cite{MugaComplex}, recupera-se a densidade de fluxo quântico.

\subsection{O problema do tempo de tunelamento}
\label{tunelamento}

 A mecânica quântica implica em uma probabilidade não nula para uma partícula transpor uma barreira de potencial maior que a energia da mesma. Prever a duração deste evento é o objetivo do estudo do tempo de tunelamento. A busca por uma definição adequada e generalizada de tempos de tunelamento para partículas massivas persiste nos dias de hoje e remonta às origens da mecânica quântica. Seu estudo começou com a descoberta do decaimento $\alpha$ no início do século 20, quando foi percebido que a partícula $\alpha$ pode escapar da barreira de potencial do núcleo, a qual seria intransponível classicamente \cite{18}.

 Atualmente, as três propostas mais tradicionais no tocante a esse problema são: o \textit{phase time} (tempo de fase) introduzido por Wigner, relacionando tempo de retardo, intervalo de interação e mudanças de fase de dispersão \cite{wigner}; o \textit{traversal time} (tempo de travessia) \cite{butlan} proposto por Buttiker e Landauer em seus estudos de tunelamento através de uma barreira modulada no tempo. Pouco tempo depois, Buttiker usou a precessão de Larmor como um relógio, identificando tempos de permanência (\textit{dwell time}), travessia e reflexão como três tempos característicos que descrevem a interação de partículas com uma barreira \cite{22}. Revisões recentes que incluem essas e outras abordagens, discutindo TOA e tempos de tunelamento de uma perspectiva moderna e unificada, podem ser encontradas em \cite{24} e \cite{25}. Outras perspectivas foram lançadas sobre essas questões através de experimentos com fótons, e revisadas em \cite{26} e \cite{21}.
