\chapter{Conclusão e considerações finais}
\label{chap:conclusao}


Nesta dissertação, revisitamos a ideia de que o problema do tempo na mecânica quântica está enraizado na objeção de Pauli, que questiona a relação de comutação entre operadores de tempo e energia. Essa objeção impulsionou o desenvolvimento de diversas abordagens para incorporar um operador tempo na mecânica quântica. No entanto, todas elas exigem o abandono da hermiticidade ou da relação de comutação com hamiltoniano. Além disso, o problema do tempo transcende a não existência de um operador auto adjunto para o tempo, pois envolve também o conceito de tempo de chegada, que não pode ser adequadamente definido no âmbito da mecânica quântica convencional.


Com esses fatos em mente, exploramos a extensão STS e as previsões de seus estados "espaço-condicionais"\text{ }(EC) para potenciais arbitrários, comparando-as com as previsões "tempo-condicionais"\text{ }da MQ usual. Vimos no Cap. \ref{chap:cap2} que o estado quântico (EC) $|\pmb{\phi}(x)\rangle$ definido em cada ponto do espaço é intrínseco à partícula, e quando expandido na base tempo, seus coeficientes representam a amplitude de probabilidade de TOA ideal na posição $x$. Ao investigarmos no Cap. \ref{chap:cap3} o comportamento da equação de autovalor do momento para um potencial arbitrário, constatamos que para potenciais dependentes do espaço, estados com momento bem definido dependem da posição, da mesma forma que estados com energia bem definida na MQ usual dependem do tempo para potenciais dependentes do tempo.


Buscamos então nesse trabalho apresentar uma interpretação clara para a equação de Schrödinger EC. Trabalhamos com a ideia que dada uma função de onda EC "inicial"\text{ }$\pmb{\phi}(t|x_0)$, a solução $\pmb{\phi}(t|x)$ é a amplitude de probabilidade da partícula chegar no instante $t$, dado que iremos agora detectar a mesma em uma nova posição $x$. Além disso, tentamos estabelecer uma relação entre os estados TC ($|\psi(t) \rangle$) e EC ($|\pmb{\phi}(x)\rangle$). Para essa comparação utilizamos amplitudes de probabilidade ${\tilde \phi}(P_b|x)$ e ${\tilde \psi}(P|t)$ da base de momento. O problema é que essas duas funções de onda representam diferentes distribuições de probabilidade: enquanto $|{\tilde \psi} (P|t)|^2$ prevê dados experimentais sobre o momento da partícula coletados em um instante fixo $t$, independentemente da posição observada, $ |{\tilde \phi}(P_b|x)|^2$ prevê dados coletados sobre o momento em uma posição fixa $x$, independentemente do tempo observado. A partir disso, ficou claro que se a extensão STS é uma teoria correta, ela deve fornecer informação complementar à MQ. Por fim, ao resolver a Eq. de Schrödinger EC para um potencial arbitrário $V = V(x)$ e aplicar para uma barreira de potencial, pudemos comparar seu comportamento com uma generalização da distribuição de Kijowski. Concluímos que as diferenças observadas entre essas distribuições pode vir do fato da extensão STS negligenciar a interferência entre os momentos positivos e negativos.



Uma perspectiva de mudança da extensão STS fornecida por esse trabalho é através do acoplamento das funções de onda EC $\phi^+ (t|x)$ e $\phi^- (t|x)$, permitindo a interferência entre momentos positivos e negativos. Pretendemos também investigar modelos operacionais do TOA (via a MQ tradicional) que ao assumir algumas idealizações dos equipamentos de medida, relatam previsões de TOA ideal previsto pela extensão STS. Por fim, uma generalização natural da extensão STS é estender suas equações para descrever o TOA em três dimensões e considerar efeitos relativísticos. 












