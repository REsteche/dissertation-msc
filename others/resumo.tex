
% resumo em português
\begin{resumo}[Resumo] 
Nós interpretamos a extensão \textit{space-time-symmetrical} (STS) da mecânica quântica (MQ) proposta em \cite{Dias} e exploramos as previsões de seus estados ``espaço-condicio\\nais'' (EC) para potenciais arbitrários. Seguindo uma quantização alternativa, onde o tempo se torna um operador auto-adjunto e a posição um parâmetro, a extensão STS postula a existência de um novo estado quântico (intrínseco à partícula), $|\pmb{\phi}(x)\rangle$, definido a cada ponto no espaço. $|\pmb{\phi}(x)\rangle$ obedece à uma equação de Schrödinger EC que, na base do tempo, prevê o tempo ideal de chegada da partícula em $x$. Neste trabalho investigamos o comportamento para um potencial arbitrário da equação de autovalor do momento, que é análoga à equação de autovalor da energia na MQ usual. Verificamos que para potenciais dependentes do espaço, estados com momento bem definido dependem da posição, assim como estados com energia bem definida na MQ usual dependem do tempo para potenciais dependentes do tempo. Posteriormente, interpretamos a equação de Schrödinger EC de forma análoga à equação de Schrödinger: Dada uma função de onda EC ``inicial'', $\pmb{\phi}(t|x_0)$, a solução $\pmb{\phi}(t|x)$ é a amplitude de probabilidade da partícula chegar no instante $t$, dado que o detector é movido para uma nova posição $x$. Neste contexto, comparando $|\psi(t) \rangle$ e $|\pmb{\phi}(x)\rangle$, os quais descrevem dados estatísticos coletados em $t$ e $x$, respectivamente, concluímos que eles fornecem informações complementares. Finalmente, resolvemos a equação Schrödinger EC para um potencial arbitrário dependente do espaço e aplicamos esta solução à uma barreira de potencial. Comparando esse resultado com uma generalização da distribuição de Kijowski, concluímos que a equação de Schrödinger EC talvez deva ser reformulada para acoplar as componentes de $\pmb{\phi}(t|x)$, levando em consideração a interferência entre momentos positivos e negativos.
% \noindent %- o resumo deve ter apenas 1 parágrafo e sem recuo de texto na primeira linha, essa tag remove o recuo. Não pode haver quebra de linha.

 \vspace{\onelineskip}
    
 \noindent
 \textbf{Palavras-chaves}: Fundamentos da Mecânica quântica. Tempo de chegada quântico. Incerteza energia-tempo. Extensão espaço-tempo-simétrica da mecânica quântica.
\end{resumo}



% resumo em inglês
\begin{resumo}[Abstract]
\begin{otherlanguage*}{english}

 %\noindent
We interpret the space-time-symmetric (STS) extension of quantum mechanics (QM) proposed in \cite{Dias} and explore the predictions of its ``space-conditional'' (SC) states for arbitrary potentials. Following an alternative quantization, where time becomes a self-adjoint operator and position a parameter, the STS extension postulates the existence of a new quantum state (intrinsic to the particle), $|\pmb{\phi}(x)\rangle$, defined at each point in space. $|\pmb{\phi}(x)\rangle$ obeys a SC Schr\"odinger equation that, in the time basis, predicts the ideal arrival time of the particle at $x$. In this work, first, we investigate for an arbitrary potential the momentum eigenvalue equation, which is analogous to the energy eigenvalue equation in the usual QM. We verify that for space-dependent potentials, states with well-defined momentum depend on position, just as states with well-defined energy in the usual QM depend on time for time-dependent potentials. Next, we interpret the SC Schr\"odinger equation analogously to the Schr\"odinger equation: Given an ``initial'' SC wave function, $\pmb{\phi}(t|x_0)$, the solution $\pmb{\phi}(t|x)$ is the probability amplitude for the particle to arrive at $t$, given that one moves the detector to a new position $x$. In this context, comparing $|\psi(t)\rangle$ and $|\pmb{\phi}(x)\rangle$, which describe statistical data at $t$ and $x$, respectively, we conclude they provide complementary information. Finally, we solve the SC Schr\"odinger equation for an arbitrary space-dependent potential and apply this solution to a potential barrier. Comparing this result with a generalization of the Kijowski distribution, We conclude that the SC Schr\"odinger equation should perhaps be reformulated to couple the components of $\pmb{\phi}(t|x)$, taking into account the interference between positive and negative momenta.



   \vspace{\onelineskip} 
 
   \noindent 
   \textbf{Keywords}: Foundations of Quantum Mechanics. Quantum arrival time. Energy-time uncertainty. Space-time-symmetric extension of quantum mechanics.
 \end{otherlanguage*}
 \end{resumo}
